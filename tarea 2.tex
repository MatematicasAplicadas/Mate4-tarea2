\documentclass{article}
\usepackage[spanish]{babel}
\usepackage[utf8]{inputenc}
\usepackage{amsmath}
\usepackage{amssymb}

\usepackage{anysize}
\marginsize{1.3cm}{1.3cm}{1.3cm}{1.3cm}

\usepackage[usenames]{color}
\definecolor{azul}{RGB}{10,80,190}

\title{
    Matemáticas para las Ciencias Aplicadas IV\\
    Tarea 02 - E.D.O. de segundo orden
}
\author{
    Careaga Carrillo Juan Manuel \\
    Quiróz Castañeda Edgar \\
    Soto Corderi Sandra del Mar
}
\date{
    03 de mayo de 2019
}
\begin{document}
    \maketitle
    {\bf Resuelve las siguientes ecuaciones diferenciales}
    \begin{enumerate}
        % Ejercicio 1
        \item {
            $\ddot y+\dot y+4y=2\senh{(t)}$\\
            \color{azul}
            
            Primero, hay que resolver ecuación homogénea asociada
            $\ddot y+\dot y+4y=0$ con polinomio característico
            $r^2+r+4$.
            Usando lo fórmula general para encontrar las raíces del polinomio
            \[
            r = \frac{-1\pm\sqrt{1^2-4(1)(4)}}{2(1)} =
            \frac{-1\pm\sqrt{15}i}{2} =
            -\frac{1}{2}\pm\frac{\sqrt{15}}{2}i
            \]
            Como son dos raíces imaginarias, entonces dos soluciones
            independientes de la ecuación
            homogénea son $e^{-\frac{t}{2}}\cos{\frac{\sqrt{15}t}{2}}$ 
            y $e^{-\frac{t}{2}}\sen{\frac{\sqrt{15}t}{2}}$, por lo que 
            la solución general de la ecuación homogénea es
            \[y = c_1e^{-\frac{t}{2}}\cos{\frac{\sqrt{15}t}{2}} + 
            c_2e^{-\frac{t}{2}}\sen{\frac{\sqrt{15}t}{2}}, \textit{ con } c_1, c_2 \in \mathbb{R}\]
            
            Ahora, hay que encontrar una solución particular de ecuación
            original. Notemos que
            \[
            \ddot y+\dot y+4y=2\senh{(t)} = 
            2(\frac{e^t-e^{-t}}{2}) = 
            e^t+ (-e^{-t})
            \]
            Por lo que usando una conjetura sensata, digamos que la
            solución particular es algo de la forma $y=Ae^t+Be^{-t}$.
            Luego, hay que calcular las derivadas.
            \begin{align*}
                \dot y &= Ae^t-Be^{-t} \\
                \ddot y &= Ae^t+Be^{-t}
            \end{align*}
            Y sustituyendo estos valores en la ecuación original, 
            \begin{align*}
                 Ae^t+Be^{-t} + Ae^t-Be^{-t} + 4Ae^t+4Be^{-t} &= 
                 e^t-e^{-t}\\
                 6Ae^x+4Be^{-t} &=e^t-e^{-t} \\
                 \implies A = \frac{1}{6} \land B = -\frac{1}{4}
            \end{align*}
            Por lo que una solución particular a la ecuación original
            es $y = \frac{e^t}{6}-\frac{e^{-t}}{4}$. \\
            Por lo tanto, la forma general de las soluciones de la 
            ecuación son 
            \[y = c_1e^{-\frac{t}{2}}\cos{\frac{\sqrt{15}t}{2}} + 
            c_2e^{-\frac{t}{2}}\sen{\frac{\sqrt{15}t}{2}} +
            \frac{e^t}{6}-\frac{e^{-t}}{4}, \text{ con } 
            c_1, c_2 \in \mathbb{R}\]\\[.03cm]
            
        }
        % Ejercicio 2
        \item {
            $2\ddot y+3\dot y+y=t^2+3\sen{(t)}$

            \color{azul}
            % Respuesta
            Primero, hay que resolver la ecuación homogénea asociada a 
            esta ecuación, que es $2\ddot y+3\dot y+y=0$, que tiene como
            polinomio característico a $2r^2+3r+1 =
            2(r+1)(r+\frac{1}{2})$.\\
            Por lo que las raíces del polinomio son $r = -1, -\frac{1}{2}$.\\
            Como son dos raíces reales, entonces dos soluciones
            independientes a la ecuación homogénea son $e^{-t}$ y 
            $e^{-\frac{t}{2}}$, por lo que la forma general de las
            soluciones es
            \[y = c_1e^{-t} + c_2e^{-\frac{t}{2}}, \text{ con }
            c_1, c_2 \in \mathbb{R}\]
            Ahora, hay que encontrar una solución particular de la
            ecuación original.
            
            La solución particular buscada será la suma de las soluciones particulares para cada
            término de la suma, es decir, $\Psi(t)=\Psi_1+\Psi_2+\Psi_3+\Psi_4$ donde
            \begin{itemize}
            	\item $\Psi_1$ es la solución particular de $2\ddot y+3\dot y+y=t^2$
            	\item $\Psi_2$ es la solución particular de $2\ddot y+3\dot y+y=3\sen{(t)}$
            \end{itemize}
            
            Resolvamos $\Psi_1$ dando una conjetura sensata, $\Psi_1=A_0t^2 + A_1t+A_2$.
            Luego, hay que calcular las derivadas.
            \begin{align*}
            \dot \Psi_1 &= 2A_0t + A_1 \\
            \ddot \Psi_1 &= 2A_0
            \end{align*}
            Y sustituyendo estos valores en la ecuación original, 
            \begin{align*}
				2(2A_0) + 3(2A_0t + A_1)+ (A_0t^2 + A_1t+A_2) &= t^2 \\
				A_0t^2 + (6A_0 + A_1)t+ (4A_0 + 3A_1 + A_2) &= t^2
            \end{align*}    	
            entonces
            \begin{equation*}
            \begin{cases}
            A_0=1 \\
            6A_0 + A_1=0 \implies A_1 = -6\\
            4A_0 + 3A_1 + A_2 = 0 \implies A_2 = 14
            \end{cases}
            \end{equation*}
            De ahí, $\Psi_1 = t^2 -6t + 14$\\
            
            Resolvamos $\Psi_2$ dando una conjetura sensata, $\Psi_2=A_0\sen(t) + A_1\cos(t)$.
            Luego, hay que calcular las derivadas.
            \begin{align*}
            	\dot \Psi_2 &= A_0\cos(t) - A_1\sen(t) \\
            	\ddot \Psi_2 &= -A_0\sen(t) - A_1\cos(t)
            \end{align*}
            Y sustituyendo estos valores en la ecuación original, 
            \begin{align*}
            2(-A_0\sen(t) - A_1\cos(t)) + 3(A_0\cos(t) - A_1\sen(t))+ (A_0\sen(t) + A_1\cos(t)) &= 3\sen{(t)} \\
            (-A_0 - 3A_1)\sen(t) + (3A_0 - A_1)\cos(t) &= 3\sen{(t)} 
            \end{align*}
            entonces
            \begin{equation*}
            \begin{cases}
            -A_0 - 3A_1=3 \\
            3A_0 - A_1=0 \implies A_1 = -\frac{9}{10} A_0 = -\frac{3}{10}
            \end{cases}
            \end{equation*}
            De ahí, $\Psi_2 = -\frac{3}{10}\sen(t) -\frac{9}{10}\cos(t)$\\
            
            Por lo tanto la solución particular es $\Psi = t^2 -6t + 14 -\frac{3}{10}\sen(t) -\frac{9}{10}\cos(t)$\\
            
            Por lo tanto, la forma general de las soluciones de la 
            ecuación son 
            \[y = c_1e^{-t} + c_2e^{-\frac{t}{2}} + t^2 -6t + 14 -\frac{3}{10}\sen(t) -\frac{9}{10}\cos(t), \text{ con } 
            c_1, c_2 \in \mathbb{R}\]
            
            
        }
        % Ejercicio 3
        \item {
            $\ddot y-6\dot y+9y=\left(3t^7-5t^4\right)e^{3t}$

            \color{azul}
            % Respuesta
            Primero, hay que encontrar la solución de la ecuación
            homogénea asociada $\ddot y-6\dot y+9y=0$.\\
            Esta tiene polinomio característico $r^2-6r+9 = (r-3)(r-3)$,
            por lo que tiene una única raíz $r = 3$.\\
            Entonces, dos soluciones independientes son $e^3t$ y
            $te^{3t}$, y todas las soluciones de la ecuación homogénea
            son de la forma
            \[y = c_1e^{3t} + c_2te^{3t}, \text{ con }
            c_1, c_2 \in \mathbb{R}\]
            Ahora hay que encontrar una solución particular de la
            ecuación original.\\
            Como conjetura sensata, digamos que una solución tendría la
            forma de $y = pe^{3t}$, con $p$ un polinomio.\\
            Derivando, tenemos que
            \begin{align*}
                \dot y &= (\dot p + 3p)e^{3t} \\
                \ddot y &= (\ddot p + 6 \dot p + 9 p)e^{3t}
            \end{align*}
            Sustituyendo en la ecuación original
            \begin{align*}
                (\ddot p + 6 \dot p + 9 p - 6 (\dot p + 3p) + 9p)e^{3t}
                &= (3t^7-5t^4)e^{3t} \\
                \ddot pe^{3t} &= (3t^7-5t^4)e^{3t}
            \end{align*}
            Y esta igualdad se da únicamente cuando $\ddot p = 3t^7-5t^4$. Entonces,
            resolviendo para $p$.
            \begin{align*}
                p = \int{\int{\ddot p dt}} = \int{\int{3t^7-5t^4 dt}} 
                = \int{(\frac{3t^8}{8}-t^5 + k_1) dt} 
                = \frac{3t^9}{72} - \frac{t^6}{6} + \frac{k_1^2}{2} + k_2
            \end{align*}
            Cada elección de $k_1$ y $k_2$ define un posible polinomio. Como sólo se
            requiere uno, se tomará el más sencillo, con $k_1 = k_2 = 0$. \\
            Entonces una solución particular de la ecuación original es 
            $y = (\frac{3t^9}{72} - \frac{t^6}{6})e^{3t}$.
            Entonces, la forma general de todas las soluciones de la ecuación
            original simplificada es 
             \[
             y = c_1e^{3t} + c_2te^{3t} + (\frac{t^9}{24} - \frac{t^6}{6})e^{3t}, 
             \text{ con } c_1, c_2 \in \mathbb{R}
            \]
        }
        % Ejercicio 4
        \item {
            $\ddot y+y=(\cos{t})(\cos{2t})(\cos{3t})$

            \color{azul}
            % Respuesta
            Primero encontremos la solución general de la ec. homogénea asociada, es decir,
            $\ddot y+y=0$, que tiene como ecuación característica $r^2+1=0$.
            \begin{align*}
                r^2+1 &= 0 \\
                r^2 &= -1 \\
                r &= \pm\sqrt{-1} \\
                r_{1,2} &= \pm i
            \end{align*}
            Como tiene raíces complejas, tomamos de una raíz su parte real como $\lambda=0$ y el
            coeficiente de su parte imaginaria como $\mu=1$ y obtenemos la solución general de 
            la homogénea asociada
            \begin{align*}
                \Phi(t) &= e^{0t}(C_1\cos{t}+C_2\sen{t}) \\
                        &= C_1\cos{t}+C_2\sen{t}
            \end{align*}
            Ahora encontremos una solución particular de la ecuación diferencial.

            Comencemos simplificando el lado derecho, utilizamos la identidad
            $\cos{u}\cos{v}=\frac{1}{2}\left[\cos{(u+v)}+\cos{(u-v)}\right]$
            \begin{align*}
                (\cos{t})(\cos{2t})(\cos{3t})
                &= (\cos{t})\frac{1}{2}\left[
                    \cos{(5t)}+\cos{(t)}
                \right]\\
                &= \frac{1}{2}\left[
                    \cos{(5t)}\cos{t}+\cos{t}\cos{t}
                \right] \\
                &= \frac{1}{2}\left[
                    \frac{1}{2}(\cos{6t}+\cos{4t})
                    +\frac{1}{2}(\cos{2t}+\cos{0})
                \right] \\
                &= \frac{1}{4}\cos{6t} + \frac{1}{4}\cos{4t} + \frac{1}{4}\cos{2t} + \frac{1}{4}
            \end{align*}
            La solución particular buscada será la suma de las soluciones particulares para cada
            término de la suma, es decir, $\Psi(t)=\Psi_1+\Psi_2+\Psi_3+\Psi_4$ donde
            \begin{itemize}
                \item $\Psi_1$ es la solución particular de $\ddot y+y=\frac{1}{4}\cos{6t}$
                \item $\Psi_2$ es la solución particular de $\ddot y+y=\frac{1}{4}\cos{4t}$
                \item $\Psi_3$ es la solución particular de $\ddot y+y=\frac{1}{4}\cos{2t}$
                \item $\Psi_4$ es la solución particular de $\ddot y+y=\frac{1}{4}$
            \end{itemize}
            Para $\ddot y+y=\frac{1}{4}$ proponemos como solución a un polinomio de grado cero, es
            decir, una constante, o sea que $\Psi_4(t)=A_0$, entonces
            $\dot\Psi_4(t)=0=\ddot\Psi_4(t)$, por lo que $A_0=\frac{1}{4}$ y por lo tanto
            $\Psi_4(t)=\frac{1}{4}$.

            Para poder proponer una solución particular de $\ddot y+y=\frac{1}{4}\cos{2t}$, primero
            resolvamos la ecuación aplicando la identidad de Euler: $\ddot y+y=\frac{1}{4}e^{2it}$.
            Notemos que $2i$ no es solución para la ecuación homogénea asociada, por lo que la
            propuesta de solución para ésta última ecuación es $\gamma(t)=A_0e^{2it}$, entonces:
            \begin{itemize}
                \item $\dot\gamma(t)=2iA_0e^{2it}$
                \item $\ddot\gamma(t)=-4A_0e^{2it}$
            \end{itemize}
            Sustituyendo tenemos que
            \begin{align*}
                -4A_0e^{2it}+A_0e^{2it} &= \frac{1}{4}e^{2it} \\[0.2cm]
                -4A_0+A_0 &= \frac{1}{4} \\[0.2cm]
                -3A_0 &= \frac{1}{4} \\[0.2cm]
                A_0 &= -\frac{1}{12}
            \end{align*}
            Por lo que $\gamma(t)=-\frac{1}{12}e^{2it}=-\frac{1}{12}[\cos{2t}+i\sen{2t}]
            =-\frac{1}{12}\cos{2t}-\frac{i}{12}\sen{2t}$. Dado que nuestra ecuación originalmente
            sólo contaba con el coseno, tomaremos únicamente la parte real de $\gamma$, por lo que
            $\Psi_3(t)=-\frac{1}{12}\cos{2t}$

            Para encontrar la solución particular de $\ddot y+y=\frac{1}{4}\cos{4t}$ podemos
            alternativamente sugerir como una propuesta sensata a $\Psi_2(t)=A\sen(4t)+B\cos(4t)$,
            de éste modo $\dot\Psi_2(t)=4A\cos{(4t)}-4B\sen{(4t)}$ y $\ddot\Psi_2(t)=-16A\sen{(4t)}
            -16B\cos{(4t)}$. Sustituyendo en $\ddot y+y=\frac{1}{4}\cos{4t}$ tenemos:
            \begin{align*}
                -16A\sen{(4t)}-16B\cos{(4t)}+A\sen(4t)+B\cos(4t) &= \frac{1}{4}\cos{(4t)} \\
                (-16A+A)\sen{(4t)}+(-16B+B)\cos{(4t)} &= 0\sen{(4t)}+\frac{1}{4}\cos{(4t)} 
            \end{align*}
            por lo que tenemos que
            \begin{equation*}
                \begin{cases}
                    -15A=0 \Rightarrow A=0 \\
                    -15B=\frac{1}{4} \Rightarrow B=-\frac{1}{60}
                \end{cases}
            \end{equation*}
            Por lo tanto $\Psi_2(t)=-\frac{1}{60}\cos{(4t)}$

            Finalmente, de manera análoga encontramos la solución particular para
            $\ddot y+y=\frac{1}{4}\cos{6t}$. Proponemos $\Psi_1(t)=A\sen{6t}+B\cos{6t}$
            \begin{itemize}
                \item $\dot\Psi_1(t)=6A\cos{6t}-6B\sen{6t}$
                \item $\ddot\Psi_1(t)=-36A\sen{6t}-36B\cos{6t}$
            \end{itemize}
            Entonces
            \begin{align*}
                -36A\sen{6t}-36B\cos{6t}+A\sen{6t}+B\cos{6t} &= \frac{1}{4}\cos{6t} \\
                -35A\sen{6t}-35B\cos{6t} &= 0\sen{6t}+\frac{1}{4}\cos{6t}
            \end{align*}
            entonces
            \begin{equation*}
                \begin{cases}
                    -35A=0 \\
                    -35B=\frac{1}{4}
                \end{cases}
            \end{equation*}
            y por lo tanto $\Psi_1(t)=-\frac{1}{140}\cos{6t}$

            Por lo tanto, la solución particular es
            $$\Psi(t)=-\frac{1}{140}\cos{6t}-\frac{1}{60}\cos{4t}-\frac{1}{12}\cos{2t}+\frac{1}{4}$$
            Y por lo tanto, la solución general se obtiene de sumar $\Phi(t)+\Psi(t)$
            \[
                y(t)
                =\Phi(t)+\Psi(t)
                = C_1\cos{t}+C_2\sen{t}
                -\frac{1}{140}\cos{6t}-\frac{1}{60}\cos{4t}-\frac{1}{12}\cos{2t}+\frac{1}{4}
            \]
        }
        % Ejercicio 5
        \item {
            $\ddot y-5\dot y+4y=t^2e^{7t}$

            \color{azul}
            % Respuesta
            Primero encontremos la solución general de la ec. homogénea asociada, es decir,
            $\ddot y-5\dot y+4y=0$, que tiene como ecuación característica $r^2-5r+4=0$.
            \begin{align*}
                r^2-5r+4 &= 0 \\
                (r-4)(r-1) &= 0 \\
                r_1-4 &= 0 \rightarrow r_1 = 4\\
                r_2-1 &= 0 \rightarrow r_2 = 1
            \end{align*}
            Cómo tiene dos raíces reales diferentes, entonces la solución general de la ecuación
            homogénea asociada es
            $$\Phi(t)=C_1e^{4t}+C_2e^{t}$$
            Ahora encontremos una solución particular de la ecuación diferencial. Como del lado
            derecho de la ecuación tenemos un polinomio de segundo grado multiplicado por una
            expresión exponencial, sería sensato suponer que una solución particular sea de la
            forma $\Psi(t)=(A_0+A_1t+A_2t^2)e^{7t}$, siempre y cuando $e^{7t}$ no sea una solución
            particular de la homogénea, pero, como puede verse, no lo es. Entonces:
            \begin{align*}
                \dot\Psi(t)
                &= (A_1+2A_2t)e^{7t}+7(A_0+A_1t+A_2t^2)e^{7t} \\
                &= \left[7A_0+A_1+(7A_1+2A_2)t+7A_2t^2\right]e^{7t}\\
                \ddot\Psi(t)
                &= [7A_1+2A_2+14A_2t]e^{7t}+7\left[7A_0+A_1+(7A_1+2A_2)t+7A_2t^2\right]e^{7t}\\
                &= \left[49A_0+14A_1+2A_2+(49A_1+28A_2)t+49A_2t^2\right]e^{7t}
            \end{align*}
            Sustituyendo en la ecuación diferencial tendremos:
            \begin{align*}
                \left[
                    49A_0+14A_1+2A_2+(49A_1+28A_2)t+49A_2t^2
                    -5(7A_0+A_1+(7A_1+2A_2)t+7A_2t^2)
                    +4(A_0+A_1t+A_2t^2)
                \right]e^{7t}\\
                 &= t^2e^{7t}
            \end{align*}
            es decir
            \begin{equation*}
                18A_0+9A_1+2A_2+(18A_1+18A_2)t+18A_2t^2=0+0\cdot t+t^2
            \end{equation*}
            por lo que
            \begin{equation*}
                \begin{cases}
                    18A_0+9A_1+2A_2 = 0 \\
                    18A_1+18A_2 = 0 \\
                    18A_2 = 1
                \end{cases}
            \end{equation*}
            Tenemos entonces que $A_2=\frac{1}{18}$, entonces $18A_1+1=0 \Rightarrow
            A_1=-\frac{1}{18}$, y entonces $18A_0-\frac{1}{2}+\frac{1}{9}=0 \Rightarrow
            A_0=\frac{7}{324}$.

            Por lo tanto la solución particular es
            $$\Psi(t)=\left(\frac{7}{324}-\frac{1}{18}t+\frac{1}{18}t^2\right)e^{7t}$$
            o bien
            $$\Psi(t)=\frac{1}{324}(18t^2-18t+7)e^{7t}$$
            Por lo que la solución general de la ecuación diferencial es $\Phi+\Psi$, que es
            $$y(t)=C_1e^{4t}+C_2e^{t}+\frac{1}{324}(18t^2-18t+7)e^{7t}$$
        }
        % Ejercicio 6
        \item {
            $\ddot y-2\dot y-3y=3te^{2t}$ con $y(0)=1$ y $\dot y(0)=0$

            \color{azul}
            % Respuesta
            Primero encontremos la solución general de la ec. homogénea asociada, es decir,
            $\ddot y-2\dot y-3y=0$, que tiene como ecuación característica $r^2-2r-3=0$, como
            $r^2-2r-3 = (r-3)(r+1) = 0$ entonces $r_1=3$ y $r_2=-1$ y la solución general de la
            homogénea asociada es:
            $$\Phi(t)=C_1e^{3t}+C_2e^{-t}$$
            Para encontrar una solución particular, proponemos como solución a
            $\Psi(t)=(A_0+A_1t)e^{2t}$, por lo que
            \begin{align*}
                \dot\Psi(t)
                &= 2(A_0+A_1t)e^{2t}+A_1e^{2t} \\
                &= (2A_0+A_1+2A_1t)e^{2t} \\
                \ddot\Psi(t)
                &= 2(2A_0+A_1+2A_1t)e^{2t}+2A_1e^{2t} \\
                &= (4A_0+4A_1+4A_1t)e^{2t}
            \end{align*}
            Sustituyendo en la ecuación diferencial tenemos
            $$(4A_0+4A_1+4A_1t)e^{2t}-2[(2A_0+A_1+2A_1t)e^{2t}]-3[(A_0+A_1t)e^{2t}]=3te^{2t}$$
            Simplificando
            $$-3A_0+2A_1-3A_1t=0+3t$$
            por lo que
            \begin{equation*}
                \begin{cases}
                    -3A_0+2A_1 = 0\\
                    -3A_1 = 3
                \end{cases}
            \end{equation*}
            Resolviendo el sistema tenemos que $A_0=-\frac{2}{3}$ y $A_1=-1$. Por lo que la
            solución particular es $\Psi(t)=(-\frac{2}{3}-t)e^{2t}$ y la solución general de la
            ecuación diferencial es 
            $$y(t)=\Phi(t)+\Psi(t)=C_1e^{3t}+C_2e^{-t}-\left(t+\frac{2}{3}\right)e^{2t}$$
            Dadas las condiciones iniciales $y(0)=1$ y $\dot y(0)=0$ y sabiendo que
            $\dot y(t)=3C_1e^{3t}-C_2e^{-t}
            -2\left(t+\frac{2}{3}\right)e^{2t}-e^{2t}
            =3C_1e^{3t}-C_2e^{-t}-(2t+\frac{7}{3})e^{2t}$
            tendremos que
            \begin{align*}
                y(0)=C_1\cdot(1)+C_2\cdot(1)-\frac{2}{3}\cdot(1) &= 1 \\
                C_1+C_2 &= \frac{5}{3} \\
                \dot y(0)= 3C_1\cdot(1)-C_2\cdot(1)-(0+\frac{7}{3})\cdot(1) &= 0 \\
                3C_1-C_2 &= \frac{7}{3}
            \end{align*}
            Sumamos ambas ecuaciones para obtener que $4C_1=\frac{12}{3}=4$, entonces $C_1=1$ y
            como $1+C_2=\frac{5}{3}$ tenemos que $C_2=\frac{2}{3}$. Por lo tanto, la solución de la
            ecuación con condiciones iniciales es
            $$y(t)=e^{3t}+\frac{2}{3}e^{-t}-\left(t+\frac{2}{3}\right)e^{2t}$$
        }
        % Ejercicio 7
        \item {
            $\ddot y+2\dot y+5y=4e^{-t}\cos{(2t)}$ con $y(0)=1$ y $\dot y(0)=0$

            \color{azul}
            % Respuesta
             Primero encontremos la solución general de la ec. homogénea asociada, es decir,
            $\ddot y+2\dot y+5y=0$, que tiene como ecuación característica $r^2+2r+5=0$.
            \begin{align*}
            r^2+2r+5 &= 0 \\
            &= \frac{-2 \pm \sqrt{4 - 4 \cdot 1 \cdot 5}}{2 \cdot 1} \\
            &= \frac{-2 \pm \sqrt{-16}}{2} \\
            &= \frac{-2 \pm 4 i}{2} \\
            &= -1 \pm 2 i
            \end{align*}
            Como tiene raíces complejas, tomamos de una raíz su parte real como $\lambda=-1$ y el
            coeficiente de su parte imaginaria como $\mu=2$ y obtenemos la solución general de 
            la homogénea asociada
            \begin{align*}
            \Phi(t) &= e^{-t}(C_1\cos{2t}+C_2 sen{2t})
            \end{align*}
            Ahora encontremos una solución particular de la ecuación diferencial.\\
            Usamos el método de la conjetura sensata, donde $\gamma(t) =  t(A_0e^{-t}sen(2t) + A_1e^{-t}cos(2t))$:
            \begin{itemize}
            	\item $\dot\gamma(t)= A_0(e^{-t}sen(2t) + t(-e^{-t}sen(2t) + 2e^{-t}cos(2t))) + A_1(e^{-t}cos(2t) + t(-e^{-t}cos(2t) - 2e^{-t}sen(2t)))  $
            	\item $\ddot\gamma(t)= A_0(-3e^{-t} tsen(2t) - 4e^{-t}tcos(2t) -2e^{-t}sin(2t) + 4e^{-t}cos(2t)) + A_1(-3e^{-t} tcos(2t) + 4e^{-t}tsen(2t) -2e^{-t}cos(2t) - 4e^{-t}sen(2t))$
            \end{itemize}
            Sustituyendo tenemos que
            \begin{align*}
                (A_0(-3e^{-t} t\sen{(2t)} - 4e^{-t}t\cos{(2t)} -2e^{-t}sen(2t) + 4e^{-t}cos(2t)) \\+ A_1(-3e^{-t} t\cos{(2t)} + 4e^{-t}t\sen{(2t)} -2e^{-t}cos(2t)- 4e^{-t}sen(2t)))\\ + 2(A_0(e^{-t}sen(2t) + t(-e^{-t}sen(2t) + 2e^{-t}cos(2t))) + A_1(e^{-t}cos(2t) + t(-e^{-t}cos(2t) - 2e^{-t}sen(2t))))\\ +   5(A_0e^{-t}t\sen{(2t)} + A_1e^{-t}t\cos{(2t)}) &= 4e^{-t}cos{(2t)} \\[0.2cm]
                4A_0e^{-t}cos{(2t)} -4A_1e^{-t}sen{(2t)}  &= 4e^{-t}\cos{(2t)}
            \end{align*}
            entonces
            \begin{equation*}
            \begin{cases}
            4A_0 =4 \\
            -4A_1 = 0
            \end{cases}
            \end{equation*}
            Por lo tanto $A_0 = 1, \ A_1 = 0$ y tenemos que la solución particular es $\gamma(t) =  e^{-t}tsen(2t)$
            
            De ahí tenemos que la solución es: $y(t) = e^{-t}(C_1\cos{2t}+C_2 sen{2t}) + e^{-t}tsen(2t)$
            
            Aplicamos las condiciones iniciales:\\
            $y(0) = e^{-0}(C_1\cos{2\cdot0}+C_2 sen{2\cdot0}) + e^{-0}\cdot0sen(2\cdot 0) = 1$\\
            $y(0) = C_1\cos{0}+C_2 sen{0} = 1$\\
            $y(0) = C_1 = 1$
            
            $y'(0) = -e^{-0}\cdot0sen(2\cdot 0) + 2e^{-0}\cdot0cos(2\cdot 0)  
            -C_2e^{-0}sen(2\cdot 0) + 2C_2e^{-0}cos(2\cdot 0)         -e^{-0}cos(2\cdot 0) -e^{-0}sen(2\cdot 0)= 0$\\
            $y'(0) = -C_2sen(0) + 2C_2cos(0) -cos(0) -sen(0)= 0$\\
            $y'(0) =  2C_2 -1 = 0$
            
            Despejamos $C_1, C_2$ y tenemos que $C_1 = 1, C_2 = \frac{1}{2}$
            
            Sustituimos $C_1, C_2$ en y(t) y tenemos de resultado final:\\
            $y(t) = e^{-t}(\cos{2t}+ \frac{1}{2}sen{2t}) + e^{-t}tsen(2t)$\\
        }
        % Ejercicio 8
        \item {
            Determinar la solución general de
            $$\ddot{y}+\lambda^2y=\sum_{m=1}^{N}{a_m sen{(m\pi t)}}$$
            con $\lambda>0$ y $\lambda\neq m\pi$ para $m=1,2,\ldots,N$.

            \color{azul}
            Primero encontremos la solución general de la ec. homogénea asociada, es decir,
            $\ddot{y}+\lambda^2y=0$, que tiene como ecuación característica $r^2+\lambda^2=0$.
            \begin{align*}
            r^2+\lambda^2 &= 0 \\
            r^2 &= -\lambda^2 \\
            r &= \pm\sqrt{-1 (\lambda^2)} \\
            Como \ \lambda > 0:
            r_{1,2} &= i\lambda
            \end{align*}
            Como tiene raíces complejas, tomamos de una raíz su parte real como $\alpha=0$ y el
            coeficiente de su parte imaginaria como $\beta=\lambda$ y obtenemos la solución general de 
            la homogénea asociada
            \begin{align*}
            \Phi(t) &= e^{0t}(C_1\cos{\lambda t}+C_2\sen{\lambda t}) \\
            &= C_1\cos{\lambda t}+C_2\sen{\lambda t}
            \end{align*}
            Ahora encontremos una solución particular de la ecuación diferencial.
            
            La solución particular buscada será la suma de las soluciones particulares para cada término de la suma, buscamos la solución particular para un k $1<k<N$ y sería la conjetura sensata: $\gamma_k = A_0 e^{k \pi i t}$
            
            \begin{itemize}
            	\item $\ddot \gamma_k(t)= A_0 (k \pi i)^2e^{k \pi i t} $
            \end{itemize}
        
         Sustituyendo tenemos que
        \begin{align*}
        \lambda^2 (A_0 e^{k \pi i t}) + (A_0 (k \pi i)^2e^{k \pi i t}) &= a_ke^{k \pi i t} \\[0.2cm]
        \end{align*}
        Entonces $A_0 (\lambda^2 + (k \pi i)^2) = a_k$ \\[0.2cm]
        De ahí, $A_0 = \frac{a_k}{\lambda^2 + (k \pi i)^2}$\\
        Sustituimos $A_0$ y tenemos que la solución particular de un k es  $\gamma_k = \frac{a_k}{\lambda^2 + (k \pi i)^2} e^{k \pi i t}$
        
        De ahí, la solución particular de toda la suma es $\gamma (t) = \sum_{k=1}^{N} \frac{a_k}{\lambda^2 + (k \pi i)^2} e^{k \pi i t} $
        
        Y tenemos que la solución final es: $y(t) = C_1\cos{\lambda t}+C_2\sen{\lambda t} + \sum_{k=1}^{N} \frac{a_k}{\lambda^2 + (k \pi i)^2} e^{k \pi i t}$ 
         
        }
    \end{enumerate}
\end{document}