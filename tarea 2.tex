\documentclass{article}
\usepackage[spanish]{babel}
\usepackage[utf8]{inputenc}
\usepackage{amsmath}

\usepackage{anysize}
\marginsize{1.3cm}{1.3cm}{1.3cm}{1.3cm}

\usepackage[usenames]{color}
\definecolor{azul}{RGB}{10,80,190}

\title{
    Matemáticas para las Ciencias Aplicadas IV\\
    Tarea 02 - E.D.O. de segundo orden
}
\author{
    Careaga Carrillo Juan Manuel \\
    Quiróz Castañeda Edgar \\
    Soto Corderi Sandra del Mar
}
\date{
    03 de mayo de 2019
}
\begin{document}
    \maketitle
    {\bf Resuelve las siguientes ecuaciones diferenciales}
    \begin{enumerate}
        % Ejercicio 1
        \item {
            $\ddot y+\dot y+4y=2\senh{(t)}$

            \color{azul}
            % Respuesta
        }
        % Ejercicio 2
        \item {
            $2\ddot y+3\dot y+y=t^2+3\sen{(t)}$

            \color{azul}
            % Respuesta
        }
        % Ejercicio 3
        \item {
            $\ddot y-6\dot y+9y=\left(3t^7-5t^4\right)e^{3t}$

            \color{azul}
            % Respuesta
        }
        % Ejercicio 4
        \item {
            $\ddot y+y=(\cos{t})(\cos{2t})(\cos{3t})$

            \color{azul}
            % Respuesta
            Primero encontremos la solución general de la ec. homogénea asociada, es decir,
            $\ddot y+y=0$, que tiene como ecuación característica $r^2+1=0$.
            \begin{align*}
                r^2+1 &= 0 \\
                r^2 &= -1 \\
                r &= \pm\sqrt{-1} \\
                r_{1,2} &= \pm i
            \end{align*}
            Como tiene raíces complejas, tomamos de una raíz su parte real como $\lambda=0$ y el
            coeficiente de su parte imaginaria como $\mu=1$ y obtenemos la solución general de 
            la homogénea asociada
            \begin{align*}
                \Phi(t) &= e^{0t}(C_1\cos{t}+C_2\sen{t}) \\
                        &= C_1\cos{t}+C_2\sen{t}
            \end{align*}
            Ahora encontremos una solución particular de la ecuación diferencial.

            Comencemos simplificando el lado derecho, utilizamos la identidad
            $\cos{u}\cos{v}=\frac{1}{2}\left[\cos{(u+v)}+\cos{(u-v)}\right]$
            \begin{align*}
                (\cos{t})(\cos{2t})(\cos{3t})
                &= (\cos{t})\frac{1}{2}\left[
                    \cos{(5t)}+\cos{(t)}
                \right]\\
                &= \frac{1}{2}\left[
                    \cos{(5t)}\cos{t}+\cos{t}\cos{t}
                \right] \\
                &= \frac{1}{2}\left[
                    \frac{1}{2}(\cos{6t}+\cos{4t})
                    +\frac{1}{2}(\cos{2t}+\cos{0})
                \right] \\
                &= \frac{1}{4}\cos{6t} + \frac{1}{4}\cos{4t} + \frac{1}{4}\cos{2t} + \frac{1}{4}
            \end{align*}
            La solución particular buscada será la suma de las soluciones particulares para cada
            término de la suma, es decir, $\Psi(t)=\Psi_1+\Psi_2+\Psi_3+\Psi_4$ donde
            \begin{itemize}
                \item $\Psi_1$ es la solución particular de $\ddot y+y=\frac{1}{4}\cos{6t}$
                \item $\Psi_2$ es la solución particular de $\ddot y+y=\frac{1}{4}\cos{4t}$
                \item $\Psi_3$ es la solución particular de $\ddot y+y=\frac{1}{4}\cos{2t}$
                \item $\Psi_4$ es la solución particular de $\ddot y+y=\frac{1}{4}$
            \end{itemize}
            Para $\ddot y+y=\frac{1}{4}$ proponemos como solución a un polinomio de grado cero, es
            decir, una constante, o sea que $\Psi_4(t)=A_0$, entonces
            $\dot\Psi_4(t)=0=\ddot\Psi_4(t)$, por lo que $A_0=\frac{1}{4}$ y por lo tanto
            $\Psi_4(t)=\frac{1}{4}$.

            Para poder proponer una solución particular de $\ddot y+y=\frac{1}{4}\cos{2t}$, primero
            resolvamos la ecuación aplicando la identidad de Euler: $\ddot y+y=\frac{1}{4}e^{2it}$.
            Notemos que $2i$ no es solución para la ecuación homogénea asociada, por lo que la
            propuesta de solución para ésta última ecuación es $\gamma(t)=A_0e^{2it}$, entonces:
            \begin{itemize}
                \item $\dot\gamma(t)=2iA_0e^{2it}$
                \item $\ddot\gamma(t)=-4A_0e^{2it}$
            \end{itemize}
            Sustituyendo tenemos que
            \begin{align*}
                -4A_0e^{2it}+A_0e^{2it} &= \frac{1}{4}e^{2it} \\[0.2cm]
                -4A_0+A_0 &= \frac{1}{4} \\[0.2cm]
                -3A_0 &= \frac{1}{4} \\[0.2cm]
                A_0 &= -\frac{1}{12}
            \end{align*}
            Por lo que $\gamma(t)=-\frac{1}{12}e^{2it}=-\frac{1}{12}[\cos{2t}+i\sen{2t}]
            =-\frac{1}{12}\cos{2t}-\frac{i}{12}\sen{2t}$. Dado que nuestra ecuación originalmente
            sólo contaba con el coseno, tomaremos únicamente la parte real de $\gamma$, por lo que
            $\Psi_3(t)=-\frac{1}{12}\cos{2t}$

            Para encontrar la solución particular de $\ddot y+y=\frac{1}{4}\cos{4t}$ podemos
            alternativamente sugerir como una propuesta sesnsata a $\Psi_2(t)=A\sen(4t)+B\cos(4t)$,
            de éste modo $\dot\Psi_2(t)=4A\cos{(4t)}-4B\sen{(4t)}$ y $\ddot\Psi_2(t)=-16A\sen{(4t)}
            -16B\cos{(4t)}$. Sustituyendo en $\ddot y+y=\frac{1}{4}\cos{4t}$ tenemos:
            \begin{align*}
                -16A\sen{(4t)}-16B\cos{(4t)}+A\sen(4t)+B\cos(4t) &= \frac{1}{4}\cos{(4t)} \\
                (-16A+A)\sen{(4t)}+(-16B+B)\cos{(4t)} &= 0\sen{(4t)}+\frac{1}{4}\cos{(4t)} \\
            \end{align*}
            por lo que tenemos que
            \begin{equation*}
                \begin{cases}
                    -15A=0 \Rightarrow A=0 \\
                    -15B=\frac{1}{4} \Rightarrow B=-\frac{1}{60}
                \end{cases}
            \end{equation*}
            Por lo tanto $\Psi_2(t)=-\frac{1}{60}\cos{(4t)}$

            Finalmente, de manera análoga encontramos la solución particular para
            $\ddot y+y=\frac{1}{4}\cos{6t}$. Proponemos $\Psi_1(t)=A\sen{6t}+B\cos{6t}$
            \begin{itemize}
                \item $\dot\Psi_1(t)=6A\cos{6t}-6B\sen{6t}$
                \item $\ddot\Psi_1(t)=-36A\sen{6t}-36B\cos{6t}$
            \end{itemize}
            Entonces
            \begin{align*}
                -36A\sen{6t}-36B\cos{6t}+A\sen{6t}+B\cos{6t} &= \frac{1}{4}\cos{6t} \\
                -35A\sen{6t}-35B\cos{6t} &= 0\sen{6t}+\frac{1}{4}\cos{6t}
            \end{align*}
            entonces
            \begin{equation*}
                \begin{cases}
                    -35A=0 \\
                    -35B=\frac{1}{4}
                \end{cases}
            \end{equation*}
            y por lo tanto $\Psi_1(t)=-\frac{1}{140}\cos{6t}$

            Por lo tanto, la solución particular es
            $$\Psi(t)=-\frac{1}{140}\cos{6t}-\frac{1}{60}\cos{4t}-\frac{1}{12}\cos{2t}+\frac{1}{4}$$
            Y por lo tanto, la solución general se obtiene de sumar $\Phi(t)+\Psi(t)$
            \[
                y(t)
                =\Phi(t)+\Psi(t)
                = C_1\cos{t}+C_2\sen{t}
                -\frac{1}{140}\cos{6t}-\frac{1}{60}\cos{4t}-\frac{1}{12}\cos{2t}+\frac{1}{4}
            \]
        }
        % Ejercicio 5
        \item {
            $\ddot y+5\dot y+4y=t^2e^{7t}$

            \color{azul}
            % Respuesta
            Primero encontremos la solución general de la ec. homogénea asociada, es decir,
            $\ddot y+5\dot y+4y=0$, que tiene como ecuación característica $r^2+5r+4=0$.
            \begin{align*}
                r^2+5r+4 &= 0 \\
                (r+4)(r+1) &= 0 \\
                r_1+4 &= 0 \rightarrow r_1 = -4\\
                r_2+1 &= 0 \rightarrow r_2 = -1
            \end{align*}
            Cómo tiene dos raíces reales diferentes, entonces la solución general de la ecuación
            homogénea asociada es
            $$\Phi(t)=C_1e^{-4t}+C_2e^{-t}$$
            Ahora encontremos una solución particular de la ecuación diferencial. Como del lado
            derecho de la ecuación tenemos un polinomio de segundo grado multiplicado por una
            expresión esxponencial, sería sensato suponer que una solución particular sea
            $\Psi(t)=v(t)e^{7t}$, siempre y cuando $e^{7t}$ no sea una solución particular de la
            homogénea, pero, como puede verse, no lo es.
        }
        % Ejercicio 6
        \item {
            $\ddot y-2\dot y-3y=3te^{2t}$ con $y(0)=1$ y $\dot y(0)=0$

            \color{azul}
            % Respuesta
            Primero encontremos la solución general de la ec. homogénea asociada, es decir,
            $\ddot y-2\dot y-3y=0$, que tiene como ecuación característica $r^2-2r-3=0$.


        }
        % Ejercicio 7
        \item {
            $\ddot y+2\dot y+5y=4e^{-t}\cos{(2t)}$ con $y(0)=1$ y $\dot y(0)=0$

            \color{azul}
            % Respuesta
        }
        % Ejercicio 8
        \item {
            Determinar la solución general de
            $$\ddot{y}+\lambda^2y=\sum_{m=1}^{N}{a_m\sen{(m\pi t)}}$$
            con $\lambda>0$ y $\lambda\neq m\pi$ para $m=1,2,\ldots,N$.

            \color{azul}
            % Respuesta
        }
    \end{enumerate}
\end{document}