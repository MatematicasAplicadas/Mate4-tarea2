\documentclass{article}
\usepackage[spanish]{babel}
\usepackage[utf8]{inputenc}

\usepackage{anysize}
\marginsize{1.3cm}{1.3cm}{1.3cm}{1.3cm}

\usepackage[usenames]{color}
\definecolor{azul}{RGB}{10,80,190}

\title{
    Matemáticas para las Ciencias Aplicadas IV\\
    Tarea 02 - E.D.O. de segundo orden
}
\author{
    Careaga Carrillo Juan Manuel \\
    Quiróz Castañeda Edgar \\
    Soto Corderi Sandra del Mar
}
\date{
    03 de mayo de 2019
}
\begin{document}
    \maketitle
    {\bf Resuelve las siguientes ecuaciones diferenciales}
    \begin{enumerate}
        % Ejercicio 1
        \item {
            $\ddot y+\dot y+4y=2\senh{(t)}$

            \color{azul}
            % Respuesta
        }
        % Ejercicio 2
        \item {
            $2\ddot y+3\dot y+y=t^2+3\sen{(t)}$

            \color{azul}
            % Respuesta
        }
        % Ejercicio 3
        \item {
            $\ddot y-6\dot y+9y=\left(3t^7-5t^4\right)e^{3t}$

            \color{azul}
            % Respuesta
        }
        % Ejercicio 4
        \item {
            $\ddot y+y=(\cos{t})(\cos{2t})(\cos{3t})$

            \color{azul}
            % Respuesta
        }
        % Ejercicio 5
        \item {
            $\ddot y+5\dot y+4y=t^2e^{7t}$

            \color{azul}
            % Respuesta
        }
        % Ejercicio 6
        \item {
            $\ddot y-2\dot y-3y=3te^{2t}$ con $y(0)=1$ y $\dot y(0)=0$

            \color{azul}
            % Respuesta
        }
        % Ejercicio 7
        \item {
            $\ddot y+2\dot y+5y=4e^{-t}\cos{(2t)}$ con $y(0)=1$ y $\dot y(0)=0$

            \color{azul}
            % Respuesta
        }
        % Ejercicio 8
        \item {
            Determinar la solución general de
            $$\ddot{y}+\lambda^2y=\sum_{m=1}^{N}{a_m\sen{(m\pi t)}}$$
            con $\lambda>0$ y $\lambda\neq m\pi$ para $m=1,2,\ldots,N$.

            \color{azul}
            % Respuesta
        }
    \end{enumerate}
\end{document}